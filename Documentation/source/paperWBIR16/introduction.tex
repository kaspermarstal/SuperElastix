\section{Introduction}
\label{sec:introduction}
% no \IEEEPARstart

The objective of image registration is to find the spatial
relationship between two or more images. Typically, intensity-based
registration methods are formulated as an optimization problem. A
transform describes the geometric mapping from one image to the
other, a metric determines the dissimilarity between the images, and
an optimizer searches for the transform giving the highest
similarity between images.

In the last decades numerous image registration methods and tools
have emerged from the research community. Implementation of these
methods, however, are scattered over a plethora of toolboxes each
with their own interface, limitations and modus operandi. It is
therefore difficult for (clinical) researchers and company
developers alike, to rigorously compare registration methods and
select the appropriate one for a given application.

The scattering is amongst others driven by the difference in
paradigms that tend to divide the field of image registration.
Opposing approaches are, for example, small deformation versus large diffeomorphic
deformation, and non-parametric versus parametric transforms. While
most software designs structure the algorithmic concepts of a
registration method in a modular fashion, i.e. having a notion of a
metric, transform or optimizer, the definition of these modules may
not be functionally compatible across paradigms. Current
registration software is typically tailored to the paradigm it
adheres, mixed with the role of mathematical definitions, data
representation and computational tricks. This severely reduces
software compatibility across toolboxes, and therefore the ability
to share components or truly compare methods rather than
implementation choices.

There are several interesting registration toolboxes that we mention
here. Arguably the greatest collection of image registration tools
can be found in the Insight ToolKit (ITK) \cite{itk}. This toolkit
implements several registration paradigms, 
most notably the parametric
intensity-based paradigm. A new framework (denoted by the extension
v4) partly supersedes the original framework and adds diffeomorphic and symmetric
registration capabilities. In addition there are the
PDEDeformableRegistration and variational framework, both implementing variations on Demons registration.
 These
paradigms, however, exhibit different object oriented software
designs, which has led to incompatibility of some of the
registration components. It is therefore difficult to develop new
algorithms that cross the borders of paradigms, as it requires a
large effort to re-factor one design into another. In addition, the
ITK provides C++ implementations rather than ready-to-use software.

Other works, using and extending ITK, are ANTs \cite{Ants}, \cite{Avants:SyN} \cite{Tustison:BSSyN}, \elastix\ \cite{elastix}, DRAMMS \cite{Ou:Dramms} and
plastimatch \cite{Sharp:plastimatch}, \cite{Shackleford:plastimatch}. 
Additionally, there are several non-ITK-based implementations including NiftyReg \cite{Modat:Niftyreg}.

These tools provide a
high-level interface (no need for programming) to configure a
registration algorithm. Many options are usually available for each
of the registration components. Rigorous evaluation to find the best
application specific configuration can, however, only be performed
within the specific paradigm or toolbox. This requires the
researcher to become acquainted with the various tools, each with
their own interface, parameters and peculiarities. Examples of such
evaluation studies include \cite{Kanai:Evaluation},
\cite{Murphy:Empire}, \cite{Klein:Evaluation},
\cite{Ou:Comparative}, 
\cite{Varadhan:Framework}.

Usability of these tools is greatly improved by environments like Nypipe \cite{Gorgolewski:Nipype} and Pydpiper \cite{Friedel:pydpiper}, which wrap each
algorithm as a module in a single python environment. This lowers
the barrier for a user as it requires dealing with only one
environment instead of multiple different environments (command
line, bash scripts, Python, Matlab, etc). A remaining disadvantage
however is that each paradigm is still treated as a monolithic block
and provides little uniformity for the detailed settings of each
algorithm. Analyzing the registration methods for a deeper
understanding of the performance differences is limited by the
ability to make their settings uniform, since the underlying
implementations are still different even for similar components. In
addition, these tools provide an interface for Python only, whereas
different users require different interfaces.

In this paper we propose a general design for unifying registration paradigms, a translation to a software design and a toolbox with an initial implementation.
Specific aims for the toolbox are i)
that the interface should be user-friendly; ii) that a single
interface can be used for a broad range of registration paradigms;
iii) that it should support algorithm modularity; which should iv)
lead to the simplification of the tuning of registration
configurations optimal for a specific task; and v) enable rigorous
comparison of registration methods rather than implementation
choices. In such an environment the user can evaluate mathematical
(or implementation) choices of specific differences between
paradigms, while eliminating the differences of other components.
Furthermore, the toolbox should provide a (research) environment
enabling exploration of cross-fertilization between paradigms.

At the core of our design there are two observations. First, even for
conceptually very different frameworks, many components are
functionally identical, albeit that those components are sometimes
used differently. And secondly, almost all registration algorithms can be
characterized by a network that organizes and connects the several
components. While some paradigms strictly split the registration
into separate modules consisting for example of a metric, a
transform and an optimizer, other paradigms combine components into
a larger block that performs the task of both the metric and a
transform. Sometimes the components are even so entangled that the
registration should be considered a single block. In other words, we
need heterogeneous levels of functionality and granularity.

In order to implement a unifying registration toolbox we propose to
reformulate registration algorithms into a software design similar
to a collaboration-based or role-based software design \cite{VanHilst:Role}, \cite{Smaragdakis:Mixin}. In
contrast to a typical object-oriented design, a registration
component is defined by what it can do (the role) instead of what it is (the class type), thereby allowing utilization of functionality across
paradigms. To this end a generic handshake mechanism is implemented
to verify whether connected components are compatible on a
mathematical as well as on a software level, thus constructing a
valid network topology. Via a single user-interface a high level
configuration is supplied, from which at run-time the corresponding
components are selected, connected into a network and executed.

In the remainder of the paper we perform an analysis of several
existing registration paradigms (Section \ref{sec:analysis}), based
on which we will propose a design for a unifying registration
toolbox (Section \ref{sec:method}). Section \ref{sec:results} then gives initial results
where we show that the toolbox is operational and allows for
specifying, running and comparing quite different registration
frameworks from a single toolbox. We end with a general discussion,
outlook and conclusion in Section \ref{sec:discussion}.

