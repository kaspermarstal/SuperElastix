\section{Discussion and Conclusions}
\label{sec:discussion}
\label{sec:conclusion}
In this paper we propose a design for a unifying toolbox based on the analysis of multiple registration paradigms.
Our design uses a role- and collaboration-based software design pattern as a framework to unify the very diverse algorithms found in literature.
This framework captures components with various degrees of granularity and functionality, as they are found in most toolboxes, and lets the user configure the network layout of the registration components.
To validate the user configuration we set up a handshake mechanism that checks whether connected components are compatible, i.e. have the correct interfaces for collaboration. After a successful configuration the registration algorithm is executed using data from user-defined sources and sinks.

Whereas this paper primarily presents the generic design that we propose, together with an initial implementation and results, the functionality of the toolbox can be greatly extended.
An immediate point of action is to extend the
functionality of the toolbox by including components from more registration paradigms, and to include more fine-grained components, such as optimizers, transforms and multi-resolution handling.
This modularization will obviously lead to more complex network graphs than currently shown in the paper.
Furthermore, we aim to include paradigms from non-ITK-based code bases, while keeping in mind component compatibility for cross fertilization of paradigms.

The toolbox is available as open source at 
\url{https://github.com/kaspermarstal/SuperElastix}
%\url{https://github.com/[anonymized]/[anonymized]},  
where the latest version can be found and where all developments can be followed.

The design we presented is both capable of generalizing across paradigms, fully supporting modularity, as well as provides an on-ramp for method integration.
This makes it a suitable foundation for a unifying registration toolbox, which, as we hope, should drastically improve the accessibility of a wide range of modern registration capabilities to a diverse audience.
