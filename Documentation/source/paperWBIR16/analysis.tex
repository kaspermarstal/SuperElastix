\section{Analysis of registration methodologies}
\label{sec:analysis}

To inventorize the requirements of the design of a unifying toolbox
we analyzed a selection of registration paradigms.
The selection was made
such that it covers some interesting variations among paradigms.
Currently, the selection excludes approaches such as discrete
optimization \cite{Glocker:Drop} \cite{Heinrich:Deeds}, groupwise registration \cite{Wachinger:Simultaneous}, geodesic
shooting \cite{Ashburner:shooting} and other Lagrangian or Hamiltonian
\cite{Vialard:AdjointShooting} methods. However, we aim for a general design
that can capture a wider span of paradigms, thus keeping
possibilities open for integration of those paradigms and additionally future
developments. The notation used in this paper is summarized in Table
\ref{tab:symbols}.


\begin{table}[b]
\centering \caption{Notation used in this paper.}\label{tab:symbols}
\begin{tabular}{lllllll}
\toprule
symbol&description\\
\midrule
$F$         & Fixed image\\
$M$         & Moving image\\
$T(\bm{x};\bm{\mu})$ & Parameterized transformation\\
$\bm{\mu}$  & Transformation parameters\\
%$\nabla\mathcal{S}$ & metric derivative\\
$\delta {\bm{\mu}}$  & parameter updates\\
$\phi$      & deformation field to moving\\
$\phi^{-1}$ & deformation field to fixed\\
$\delta v$  & forces update field\\
%$\phi_i$    & deformation field to virtual\\
$S_\phi()$, $S_v()$  & diffusion-like regularization\\
$S_\delta()$& fluid-like regularization\\
$v_t$       & time varying velocity field\\
%$V$         & Virtual (in between image)\\
%$v_i$       & (greedy) update velocity field\\
$v_\mathrm{stat}$   & stationary velocity field\\
$\mathrm{BCH}$  & Baker-Campbell-Hausdorff update rule\\% for $v_\mathrm{stat}$\\%
%&Moving gradient&&&$-\nabla (M\circ\phi)(x)$&-&\\
%&symmetric gradient&&&\scalebox{0.5}{$-\frac{(\nabla F(x) + \nabla (M \circ \phi)(x))}{2}$}&\scalebox{0.5}{$\frac{(\nabla F(x) + \nabla (M \circ \phi)(x))}{2}$}&\\
%&Thirion's gradient&&&$-\nabla F(x)$&&\\       \bottomrule
\end{tabular}
\end{table}

\subsection{Existing methods}

The majority of established registration algorithms exhibits distinct
conceptual components each performing a sub-task in the registration
procedure. Across toolboxes and paradigms, however, these components
can vary in functionality and granularity. Some toolboxes are
subdivided into many small components, while others consist of a few
larger components. These components can be part of multiple or only
a single paradigm. Even if components implement the same concept
they may not be interchangeable due to mathematical or software
differences.

To expose the typical variation among toolboxes and paradigms, we
analyzed 5 registration methods in detail, namely: B-spline
registration \cite{elastix}, log-diffeomorphic Demons \cite{Ashburner:Dartel}, time varying velocity field
registration \cite{Tustison:BSSyN}, symmetric log-diffeomorphic Demons \cite{Vercauteren:SymLogDemons} and greedy
Symmetric Normalization (gSyN) \cite{Avants:SyN}.
We created detailed networks to facilitate analysis of these
paradigms, see Figure \ref{fig:networkanalysis}.

The first example, as illustrated in Figure \ref{fig:bspline}, is a
registration with a parametric transform, e.g. a B-spline transform
which is parameterized by coefficients $\bm{\mu}$.
The parameters $\bm{\mu}$ span a Euclidean space of possible
solutions in which the optimizer searches for the optimal image
alignment. The assumption of a Euclidean optimization space versus a
manifold (subspace) is fundamental for many optimization strategies.
For example, a large group of optimization strategies performs
incremental updates of the form $\bm{\mu}^{k+1} = \bm{\mu}^k +
\bm{\mu}_\delta^k$, amongst others (unconstrained) gradient descent
routines, such as conjugate gradient, adaptive stochastic gradient
descent, or quasi-Newton methods.

Variations on Demons registration, such as diffeomorphic Demons (not
illustrated) and log-diffeomorphic Demons \cite{Dru:LogDemons}
(Figure \ref{fig:LogDemons}) introduce, amongst others, different
update rules. Instead of the incremental update $\phi^{k+1} = \phi^k
+ \phi_\delta^k$ in classical Demons, a functional composition
$\phi^{k+1} = \phi^k \circ \phi_\delta^k$ is used in diffeomorphic
Demons. Another difference is that where diffeomorphic Demons models
the transformation by a vector field pointing from one image to the
other, the log-diffeomorphic Demons models such transformations by a
stationary velocity field. Instead of an additive or compositional
update log-diffeomorphic Demons uses the function
$v_\mathrm{stat}^{k+1} = \mathrm{BCH}\left(v_\mathrm{stat}^{k},
\phi_\delta^k\right)$. 

While the velocity field of the log-diffeomorphic Demons transform
is a stationary vector field, LDDMM-based approaches \cite{Beg:LDDMM}, see Figure
\ref{fig:timevarying}, use a time varying velocity field. The
modeling of a (virtual) time results in an extra dimension on the
velocity field, which can be expressed by parallel paths in the
network layout.

Various registration methods focus on symmetry of the mapping
process, so that interchanging the role of the fixed and moving
image does not produce different results. Examples of these methods
are the symmetric version of log-Demons, see Figure
\ref{fig:SymLogDemons} and greedy SyN (gSyN) \cite{Avants:SyN}, see
Figure \ref{fig:SyN}. The symmetry imposed by registration methods
is typically reflected in the network layout. Whereas gSyN consists
of two fully (anti) symmetric paths, the network of symmetric
log-Demons exposes only a partial symmetry.

\begin{figure*}[!tb]
\centering
\subfloat[B-spline registration]{
  \scalebox{0.6}{\input{bsplines5.pdf_tex}}
  \label{fig:bspline}
}
\hfil
\subfloat[Log-diffeomorphic Demons]{
  \scalebox{0.6}{\input{logdem5.pdf_tex}}
  \label{fig:LogDemons}
} \hfil \subfloat[ANTs time varying velocity field registration.
ANTs' interpretation of LDDMM.]{
  \scalebox{0.6}{\input{timevarying5.pdf_tex}}
  \label{fig:timevarying}
}
\hfil
\subfloat[Symmetric log-diffeomorphic Demons]{
  \scalebox{0.6}{\input{symlogdem5.pdf_tex}}
  \label{fig:SymLogDemons}
}
\hfil
\subfloat[Greedy Symmetric Normalization (gSyN)]{
  \scalebox{0.6}{\input{syn5.pdf_tex}}
  \label{fig:SyN}
} \caption{Networks of five exemplary registration
frameworks.}\label{fig:networkanalysis}
\end{figure*}

\subsection{Analysis}

Generalizing from the discussion above, in order to come to a
unifying design, we observe the following. Registration algorithms
can generally be described as networks of functional components. The
network layout of the various registration paradigms, however, are
considerably different. Some networks may be summarized as a
pipeline of components, whereas others may consist of parallel paths
(e.g. symmetric Demons and gSyN) with more or less
inter-connectivity between the components. Among paradigms,
components of a network can be present in other networks as well,
although possibly at different locations.

Components also vary in how exchangeable they are between paradigms.
Some components are completely paradigm specific, such as the inversion component from gSyN in Figure \ref{fig:SyN}. Other
components can be freely exchanged between paradigms, for example
image blurring components for multi-resolution approaches (not shown). In between there is a gray area of components
that seem similar task-wise, but due to mathematical differences
between paradigms the components are not straightforwardly
interchangeable.
Examples are in the group of metric components: Demons metrics generally include dedicated terms for the ESM optimization routine which are not present in B-spline registration, and B-spline methods integrally perform a mapping of the metric gradient to the B-spline transform parameters.
instead of providing a vector field like in Demons and other diffeomorphic methods.
This difference can have a
software-related origin as well, for instance when the type of data
communicated between components differs, or when toolboxes have a slightly different definition of the boundaries of a component.

The degree of granularity of paradigms can be different as well.
For example in parametric (e.g. B-spline) registration the optimization strategy is typically regarded as a plug-in component, which can be chosen independently from the transformation model, whereas in gSyN the optimization strategy is inherent to the registration paradigm.
To regard components as potentially being composed of multiple tasks has a practical advantage as well, as it would enable
integration of complete toolboxes coarse-grained or even monolithically.

To conclude, a design that can capture a broad variety of paradigms
1) needs to be very flexible with respect to the algorithmic network
layout and 2) needs to be able to capture a wide variation 
of components with various levels of granularity. We therefore propose to follow a
collaboration-based or role-based software design pattern
\cite{Smaragdakis:Mixin}, in which the definition of a component is
defined by the role(s) it can take, rather than by its inheritance,
as is used in all current major registration toolboxes.
